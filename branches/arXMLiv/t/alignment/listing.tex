\documentclass{article}
\usepackage{makeidx}
\makeindex
\usepackage{listings}
\usepackage{color}
\begin{document}
\lstset{numbers=left}

\section{Introduction}
This document contains the following listings:
\lstlistoflistings

\section{Inline Listings}
Various delimiters: \lstinline{a_word},
\lstinline!a_word!, \lstinline Aa_wordA,
\lstinline&a_word& and even \lstinline^a_word^ done.

\section{A Pascal Listing}
A listing portion:
\begin{lstlisting}[language=Pascal,firstline=2,lastline=5,caption={}]
for i:=maxint to 0 do
begin
  { do nothing }
end;

Write('case insensitive');
Write('long '' string');
WritE('Pascal keywords.');
\end{lstlisting}

A numbered listing:
\begin{lstlisting}[language=Pascal,numbers=left, numberstyle=\tiny, stepnumber=2,stringstyle=\ttfamily,showspaces,tabsize=4]
for i:=maxint to 0 do
	begin
		{ do nothing }
	end;

Write('case insensitive');
Write('long '' string');
WritE('Pascal keywords.');
\end{lstlisting}

A Titled listing:
\begin{lstlisting}[language=Pascal,title={A bit of Pascal}]
for i:=maxint to 0 do
begin
  { do nothing }
end;
Write('case insensitive');
\end{lstlisting}


A Captioned listing (known as Listing \ref{pascallisting}) :
\begin{lstlisting}[language=Pascal,caption=Another bit of Pascal, label=pascallisting,firstnumber=100,numbers=left]
for i:=maxint to 0 do
begin
  { do nothing }
end;
\end{lstlisting}

\section{An Environment}
\begin{lstlisting}[language=Pascal]
for i:=maxint to 0 do
begin
  { do nothing }
end;
\end{lstlisting}

\lstnewenvironment{colored}[1]{\lstset{language=Pascal,numbers=left,numberstyle=\color{#1}}}{}
\begin{colored}{red}
for i:=maxint to 0 do
begin
  { do nothing }
end;
\end{colored}

\begin{colored}{blue}
for i:=maxint to 0 do
begin
  { do nothing }
end;
\end{colored}

\section{Listing with Math}
\begin{lstlisting}[language=c,texcl]
// \upshape calculate $a_{ij}$
a[i][j] = a[j][j]/a[i][j];
\end{lstlisting}

\begin{lstlisting}[texcl,language=c]
// \upshape calculate $a_{ij}$
a[i][j] = a[j][j]/a[i][j];
\end{lstlisting}

\begin{lstlisting}[language=c,mathescape,numbers=left]
// calculate $a_{ij}$
$a_{ij}
 = a_{jj}/a{ij}$;
// calculate $a_{ij} = 
\sin x$
a[i,j]=sin(x)
foo="a word";
foo="a $x^2$ math";
\end{lstlisting}

\begin{lstlisting}[language=c,escapechar=\%,escapebegin=\textless,escapeend=\textgreater,numbers=left]
// calculate %$a_{ij}$%
a_{ij}
 = a_{jj}/a{ij};
\end{lstlisting}

\begin{lstlisting}[language=c,numbers=left,stringstyle=\ttfamily]
// calculate $a_{ij}$
$a_{ij}
 = a_{jj}/a{ij}$;
// calculate $a_{ij} = 
\sin x$
a[i,j]=sin(x)
foo="a word";
foo="a \"string";
foo="a $x^2$ math";
\end{lstlisting}

\section{A Perl Listing}
\lstinputlisting[language=perl]{any.sty.ltxml}

\section{A Recursive \TeX\ listing}
\lstinputlisting[language={[LaTeX]TeX}]{listing.tex}

\section{Testing Tag}
% AHA, tagstyle only is in effect with XML (?)
\begin{lstlisting}[language=XML,tagstyle=\bf]
<element attr='value'>content</element>
\end{lstlisting}
\begin{lstlisting}[language=XML,tagstyle=\bf,usekeywordsintag=false]
<element attr='value'>content</element>
\end{lstlisting}
\begin{lstlisting}[language=XML,tagstyle=\bf,markfirstintag]
<element attr='value'>content</element>
\end{lstlisting}

\section{Screwiness}
\lstdefinelanguage{bingo}{morekeywords={foo,bar},morekeywords=[2]{bing,bar}}
%,
% AHA, words can only be in one class (1st one declared?)
% BUT, index is separate, and classname is without the "style" !!
\begin{lstlisting}[language=bingo,keywordstyle=\bfseries,keywordstyle={[2]\itshape},index={[1][keywords2]{bar,baz}}]
foo bar baz bing booboo
\end{lstlisting}
{\bfseries\itshape bfit}
{\itshape\bfseries itbf}
\printindex
\end{document}
