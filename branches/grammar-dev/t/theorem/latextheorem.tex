\documentclass{article}
% Simplified extract of the tests of ams theorem.
\title{\LaTeX's Newtheorem}
\date{none}
\newtheorem{thm}{Theorem}[section]
\newtheorem{cor}[thm]{Corollary}
\newtheorem{prop}{Proposition}
\newtheorem{lem}[thm]{Lemma}

%    Because the amsmath pkg is not used, we need to define a couple of
%    commands in more primitive terms.
\let\lvert=|\let\rvert=|
\newcommand{\Ric}{\mathop{\mathrm{Ric}}\nolimits}

%    Dispel annoying problem of slightly overlong lines:
\addtolength{\textwidth}{8pt}

\begin{document}
\maketitle

\section{Test of standard theorem styles}

\begin{lem}[negatively curved families]
Let $\{ds_1^2,\dots,ds_k^2\}$ be a negatively curved family of metrics
on $\mathbf{D}_r$, with associated forms $\omega^1$, \dots, $\omega^k$.
Then $\omega^i \leq\omega_r$ for all $i$.
\end{lem}

Then our main theorem:
\begin{thm}\label{pigspan}
Let $d_{\max}$ and $d_{\min}$ be the maximum, resp.\ minimum distance
between any two adjacent vertices of a quadrilateral $Q$. Let $\sigma$
be the diagonal pigspan of a pig $P$ with four legs.
Then $P$ is capable of standing on the corners of $Q$ iff
\begin{equation}\label{sdq}
\sigma\geq \sqrt{d_{\max}^2+d_{\min}^2}.
\end{equation}
\end{thm}

\begin{cor}
Admitting reflection and rotation, a three-legged pig $P$ is capable of
standing on the corners of a triangle $T$ iff (\ref{sdq}) holds.
\end{cor}

\end{document}
