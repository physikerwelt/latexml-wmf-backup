\documentclass{article}
\newcount\num
\chardef\chara='3\relax
\mathchardef\mchara"1350        %= 4944
\begin{document}
\section{chardef}
% Regular counter register.
% use the CS alone means assignment
\num=10\relax
% use with \the to get the numerical value
10 = \the\num

% Chardef register.
% use the CS alone gets the character
Lambda = \chara

% Use with \the to get the numerical value
3=\the\chara

% However, on RHS of assignment, without \the
\num=\chara
3=\the\num

\section{mathchardef}
% What about mathchar ?
Sum = $\mchara$

\num=\mchara
4944=\the\num

4944=\the\mchara

% AND the perverse use of chardefs for plain number constants
\makeatletter
10000 = \the\@M

\num=\@M
10000 = \the\num

\section{fontencoding}
% Note that chardef'd chars see the current font encoding

OML: \fontencoding{OML}\selectfont\chara\fontencoding{OT1}\selectfont.

OMS: \fontencoding{OMS}\selectfont\chara\fontencoding{OT1}\selectfont.

OMX: \fontencoding{OMX}\selectfont\chara\fontencoding{OT1}\selectfont.

% But mathchardef'd chars have it effectively built in (via font family)

OML: $\fontencoding{OML}\selectfont\mchara\fontencoding{OT1}\selectfont$.

OMS: $\fontencoding{OMS}\selectfont\mchara\fontencoding{OT1}\selectfont$.

OMX: $\fontencoding{OMX}\selectfont\mchara\fontencoding{OT1}\selectfont$.

\end{document}
