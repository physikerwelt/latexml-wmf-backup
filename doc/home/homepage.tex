\documentclass{article}
\usepackage{latexml}
\usepackage{alltt}
\usepackage{hyperref}
\usepackage{../sty/latexmldoc}
\newcommand{\current}{0.5.99}
\title{\LaTeXML\ \emph{A \LaTeX\ to XML Converter}}
%\toctitle{A \LaTeX\ to XML Converter}
\begin{document}
\maketitle

%============================================================
The \emph{Current release} is \htmlref{\current}{download}.

In the process of developing the
\href{http://dlmf.nist.gov/}{Digital Library of Mathematical Functions},
we needed a means of transforming
the LaTeX sources of our material into XML which would be used
for further manipulations, rearrangements and construction of the web site.
In particular, a true `Digital Library' should focus on the \emph{semantics}
of the material, and so we should convert the mathematical material into both
content and presentation MathML.
At the time, we found no software suitable to our needs, so we began
development of LaTeXML in-house.  

In brief, \texttt{latexml} is a program, written in Perl, that attempts to
faithfully mimic \TeX's behaviour, but produces XML instead of dvi.
The document model of the target XML makes explicit the model implied
by \LaTeX.
The processing and model are both extensible; you can define
the mapping between \TeX\ constructs and the XML fragments to be created.
A postprocessor, \texttt{latexmlpost} converts this
XML into other formats such as HTML or XHTML, with options
to convert the math into MathML (currently only presentation) or images.

\emph{Caveats} This is still a \emph{preview release} of LaTeXML.
It isn't finished, there are large gaps in the coverage,
particularly in missing implementations of the many useful \LaTeX\ packages.
But is beginning to stabilize and interested parties
are invited to try it out, give feedback and even to help out.


%============================================================
\section{Examples}\label{examples}
At the moment, the best example of \LaTeXML's output is the DLMF itself,
containing the Airy Functions chapter.
It is perhaps not a fair example, in that there is much
special-purpose processing to generate cross-referencing and metadata that is not
part of LaTeXML by default. Nevertheless, it gives an indication of the
math conversion and other features possible.  A significantly less
challenging example is the LaTeXML overview document.

\begin{description}
\item[\href{http://dlmf.nist.gov/contents/}{DLMF}]
   The Digitial Library of Mathematical Functions was the
   primary instigator for this project.
\item[\href{manual/}{LaTeXML Manual}]
   The \LaTeXML\ User's manual.
\item[\href{examples/tabular/tabular.xhtml}{LaTeX tabular}]
    from the \LaTeX\ manual, p.205.
    (\href{examples/tabular/tabular.tex}{\TeX},
     \href{examples/tabular/tabular.dvi}{dvi})
\item[These pages] were produced using \LaTeXML, as well.
\end{description}

\section{Download}\label{download}
\paragraph{Current Release}
\href{LaTeXML-\current.tar.gz}{LaTeXML-\current.tar.gz};
See also the \href{Changes}{Change Log}.

\paragraph{SVN Repository}
Browsable at \url{https://svn.mathweb.org/repos/LaTeXML}.
Anonymous checkout is available using the command:
\begin{alltt}
  svn co https://svn.mathweb.org/repos/LaTeXML
\end{alltt}

\paragraph{Archive}
\href{archive/LaTeXML-0.5.1.tar.gz}{0.5.9},
\href{archive/LaTeXML-0.5.1.tar.gz}{0.5.1},
\href{archive/LaTeXML-0.5.0.tar.gz}{0.5.0},
\href{archive/LaTeXML-0.4.1.tar.gz}{0.4.1},
\href{archive/LaTeXML-0.4.0.tar.gz}{0.4.0},
\href{archive/LaTeXML-0.3.2.tar.gz}{0.3.2},
\href{archive/LaTeXML-0.3.1.tar.gz}{0.3.1},
\href{archive/LaTeXML-0.3.0.tar.gz}{0.3.0},
\href{archive/LaTeXML-0.2.99.tar.gz}{0.2.99},
\href{archive/LaTeXML-0.2.2.tar.gz}{0.2.2},
\href{archive/LaTeXML-0.2.1.tar.gz}{0.2.2},
\href{archive/LaTeXML-0.2.0.tar.gz}{0.2.0},
\href{archive/LaTeXML-0.1.2.tar.gz}{0.1.2},
\href{archive/LaTeXML-0.1.1.tar.gz}{0.1.1},
\href{archive/LaTeXML-0.1.0.tar.gz}{0.1.0}.

\paragraph{Licence}
As this software was developed as part of work done by the
United States Government, it is not subject to copyright,
and is in the public domain.
Note that according to
\href{http://www.gnu.org/licences/license-list.html#PublicDomain}{Gnu.org}
public domain is compatible with GPL.

%============================================================
\section{Installation}\label{install}

\LaTeXML\ itself is written in `pure' Perl, but requires several
perl modules available from CPAN or your OS distribution.
Some of the modules provide Perl bindings to libraries
written in C or C++, and thus require those libraries to have been installed.
A recent \href{http://www.perl.org/}{Perl}
with sufficient Unicode support is required: version 5.8 or later.

\subsection{Prerequisites}

% %3A%3A for :: ???
\begin{description}
\item[\href{http://search.cpan.org/search?query=Parse::RecDescent&mode=module}{Parse::RecDescent}]
    A useful grammar based parser module.
\item[\href{http://search.cpan.org/search?query=Image::Magick&mode=module}{Image::Magick}]
    This module provides bindings to the \href{http://www.imagemagick.org/}{ImageMagick} library.
\item[\href{http://search.cpan.org/search?query=XML::LibXML&mode=module}{XML::LibXML}]
    This module provides bindings to the C library libxml2
    (available from \href{http://www.xmlsoft.org}{xmlsoft}.
    At least version 1.61 is required, unless the separate \texttt{XML::LibXML::XPathContext}
    package is installed (but it is not standard in many distributions).
\item[\href{http://search.cpan.org/search?query=XML::LibXSLT&mode=module}{XML::LibXSLT}]
    This module provides bindings to the C library libxslt
    (from \href{http://www.xmlsoft.org}{xmlsoft},
%  <li><tt><a href="http://search.cpan.org/search?query=XML::LibXML::XPathContext&mode=module">XML::LibXML::XPathContext</a></tt>
%      Depends on the XML::LibXML module.
\item[\href{http://search.cpan.org/search?query=DB_File&mode=module}{DB\_File}]
    This module is usually part of a standard Perl installation, provided
    \href{http://www.sleepycat.com}{BerkeleyDB} is installed.
\item[\href{http://search.cpan.org/search?query=Test::Simple&mode=module}{Test::Simple}]
    This module is usually(?) part of a standard Perl installation.
\end{description}

\paragraph{CPAN}
Assuming that all the required C packages are already installed on your system,
  the following command should be sufficient to install the required perl 
  modules (typically as root):
\begin{alltt}
   perl -MCPAN -e shell
   install Parse::RecDescent, XML::LibXML, XML::LibXSLT, DB_File
   quit
\end{alltt}
The \texttt{Image::Magick} module often installs best by enabling the
perl bindings as part of compiling and installing the main ImageMagick package.

\paragraph[RPM-based systems]{RPM-based systems (Fedora, RedHat, CentOS)}
For Fedora, RedHat, or similar rpm-based distributions,
the prerequisites (and thier dependencies) can be installed using
(as root):
\begin{alltt}
   yum install perl-Parse-RecDescent  \textbackslash\\
     ImageMagick ImageMagick-perl  \textbackslash\\
     libxml2 perl-XML-LibXML libxslt perl-XML-LibXSLT \textbackslash\\
     perl-Test-Simple
\end{alltt}

\paragraph{Debian-based systems}
For Debian based distributions, presumably including Ubuntu, the following
command  should install the prerequisites:
\begin{alltt}
   sudo apt-get install   \textbackslash\\
      libparse-recdescent-perl \textbackslash\\
      libxml2 libxml-libxml-perl \textbackslash\\
      libxslt1 libxml-libxslt-perl  \textbackslash\\
      imagemagick perlmagick
\end{alltt}
Some \href{http://rhaptos.org/devblog/reedstrm/latexml}{notes} on installation on Debian
based systems are also available.

\paragraph{MacOS}
To be done.

\paragraph{Windows}
To be done.

\subsection{Installing}
Installation of \LaTeXML\ from source is via the usual Perl module incantations:
\begin{alltt}
   tar zxvf latexml-\current.tar.gz
   cd latexml-\current\\
   perl Makefile.PL
   make
   make test
\end{alltt}
and then, as root:
\begin{alltt}
   make install
\end{alltt}
(See \texttt{perl perlmodinstall} for more details, if needed.)

I hope, in the near future, to be able to provide precompiled packages
for at least some of the above OS's.

%============================================================
\section{Documentation}\label{docs}
If you're lucky, all that should be needed to convert
a \TeX\ file, \textit{mydoc}\texttt{.tex} to XML, and
then to XHTML+MathML would be:
\begin{alltt}
   latexml --dest=\textit{mydoc}.xml \textit{mydoc}
   latexmlpost -dest=\textit{somewhere/mydoc}.xhtml \textit{mydoc}.xml
\end{alltt}
This will carry out a default transformation into XHTML+MathML.  If you
give the destination extension with html, it will generate HTML+images.

If you're not so lucky, or want to get fancy, well \ldots dig deeper:
\begin{description}
\item[\href{manual/index.xhtml}{LaTeXML Manual}]
    Overview of LaTeXML.
\item[\href{manual/commands/latexml.xhtml}{\texttt{latexml}}]
    describes the \texttt{latexml} command.
\item[\href{manual/commands/latexmlpost.xhtml}{\texttt{latexmlpost} command}]
   describes the \texttt{latexmlpost} command for postprocessing.
\end{description}

% Possibly, eventually, want to expose:
%   http://www.mathweb.org/wiki/????
% But, it doesn't have anything in it yet.

%============================================================
\section{Contacts \& Support}\label{contact}

\paragraph{Mailing List}
There is a low-volume mailing list for questions, support and comments.
See \href{http://lists.jacobs-university.de/mailman/listinfo/project-latexml}{\texttt{latexml-project}} for subscription information.


\paragraph{Bug-Tracker}
  There is a Trac bug-tracking system for reporting bugs, or checking the
  status of previously reported bugs at
 \href{https://trac.mathweb.org/LaTeXML/}{Bug-Tracker}.


\paragraph{Thanks} to our friends at
the \href{http://kwarc.info}{KWARC Research Group}
for hosting the mailing list, Trac system and svn repository,
as well as general moral support.
Thanks also to \href{http://nist.gov/sima}{Systems Integration for Manufacturing Applications}
for funding portions of the research and development.

\paragraph{Author} \href{mailto:bruce.miller@nist.gov}{Bruce Miller}.
%============================================================
\section{Privacy Notice}\label{privacy}
We adhere to \href{http://www.nist.gov/public_affairs/privacy.htm}{NIST's Privacy, Security and Accessibility Policy}.
%============================================================

\end{document}