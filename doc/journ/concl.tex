\section{Conclusion and Future Work}\label{sec:concl}

We have presented a system {\stex} of macro-packages for semantic annotation of
{\TeX/\LaTeX} documents and an extension of the {\latexml} system with bindings
for the {\stex} package. This allows us to semantically pre-load {\TeX/\LaTeX}
documents and transform them into {\xml} and ultimately into the {\omdoc} format.

The system is being tested on a first-year computer science course at
International University Bremen. The next case study will be the {\omdoc} 1.2
report~\cite{Kohlhase:omdoc1.2}.

In essence, the {\stex} package together with its {\latexml} bindings forms an
invasive editor for {\omdoc} in the sense discussed in~\cite{KohKoh:cdad04}: The
author can stay in her accustomed work-flow; in the case of {\TeX/\LaTeX}, she can
use the preferred text editor to ``program'' documents consisting of text,
formulae, and control sequences for macros. The documents can even be presented in
by the {\TeX} formatter in the usual way. Only with the semantic preloading, they
can be interpreted as MKM formats that contain the necessary semantic information,
and can even be transformed into explicit MKM formats like {\omdoc}.

Thus the {\stex}/{\latexml} combination extends the available invasive editors
for {\omdoc} to three ({\cpoint}~\cite{KohKoh:cdad04} and
{\sc{nb2omdoc}}~\cite{Sutner:cmnto04} being those for {\ppt} and {\mathematica}).
This covers the paradigmatic examples of scientific document creation formats
(with the exception of MS Word a possible porting target of the VBA-based
application {\cpoint}).

In the future, we plan to extend the system with a {\cnxml} back-end, which
produces the input format for {\connexions} and e-learning document management
system at Rice University. In particular, we will include the {\LaTeX}
package~\cite{WillHenBar:clsfflc03} developed there.


\subsection*{Acknowledgments} This work has profited significantly from
discussions with Bruce Miller and Ioan Sucan. The former is the author of the
{\latexml} system and has extended his system readily to meet the new demands from
our project. The latter is a student a IUB who has faithfully carried the brunt of
the editing load involved with semantic pre-loading the {\LaTeX} slides. Finally I
am indebted to David Carlisle who helped me with the non-trivial hacking involved
in getting the modules to work.

\begin{oldpart}{another conclusion, use parts of this}
\subsection{Conclusion}\label{sec:concl}

The {\stex} collection provides a set of semantic macros that extends the familiar and
time-tried {\LaTeX} workflow in academics until the last step of Internet publication of
the material. For instance, a {\connexions} module can be authored and maintained in
{\LaTeX} using a simple text editor, a process most academics in technical subjects are
well familiar with. Only in a last publishing step (which is fully automatic) does it get
transformed into the {\xml} world, which is unfamiliar to most academics. 

Thus, {\stex} can serve as a conceptual interface between the document author and MKM
systems: Technically, the semantically preloaded {\LaTeX} documents are transformed into
the (usually {\xml}-based) MKM representation formats, but conceptually, the ability to
semantically annotate the source document is sufficient.

 
The {\stex} macro packages have been validated together with a case
study~\cite{Kohlhase:smtl05}, where we semantically preload the course materials for a
two-semester course in Computer Science at International University Bremen and transform
them to the {\omdoc} MKM format, so that they can be used in the {\activemath}
system~\cite{activemathAIEDJ01}. Another study of converting {\LaTeX} materials for the
{\connexions} project is under way.\ednote{say some more}
\end{oldpart}

\begin{oldpart}{use somewhere}
  The nice thing about the infrastructure is that you can still run {\LaTeX} over the
  first form and get the same formula in the DVI file that you would have gotten from
  running it over the second form. That means, if the author is prepared to write the
  mathematical formulae a little differently in her {\LaTeX} sources, then she can use
  them in {\xml} and {\LaTeX} at the same time.
\end{oldpart}

%%% Local Variables: 
%%% mode: stex
%%% TeX-master: "main"
%%% End: 
