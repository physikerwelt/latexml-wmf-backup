\documentclass{article}
\usepackage[show]{ed,draftstamp}
\usepackage{a4wide,url,amssymb,myref,moreverb}
\usepackage{stex} % assuming that you have sTeX styles on the TEXINPUTS
\def\blue{\bf}\def\cM{{\cal M}}
\def\phi{\varphi}
\def\lec#1{\hfill(#1)}
\def\interpret#1#2#3{{\left[\kern-0.18em\left[#1\right]\kern-0.18em\right]^{#2}_{#3}}}
\def\intermp#1{\interpret{#1}{{\cal M}}{\phi}}\def\interoo#1{\interpret{#1}{}{}}
\def\interm#1{\interpret{#1}{\cM}{}}\def\interp#1{\interpret{#1}{}{\phi}}

%\usepackage[savemem]{listings} 
%\usepackage{lstpatch}
\usepackage{graphicx,hyperref}
\usepackage{float}
\floatstyle{boxed}
\newfloat{exfig}{thp}{lop}
\floatname{exfig}{Example}

\def\twindef#1#2{{\defemph{\twintoo{#1}{#2}}}}
\def\twintoo#1#2{{#1} {#2}}  %mock index
\def\twin#1#2{}%mock index
\def\defemph#1{{\bf{#1}}}
\def\ttin#1{{\tt{#1}}}
\def\defin#1{{\defemph{#1}}}

\def\sys#1{{{#1}}\index{#1@{#1}}}
\def\scsys#1{{{\sc #1}}\index{#1@{\sc #1}}}
\def\hermes{\scsys{Hermes}}
\def\xml{\scsys{xml}}
\def\ppt{\scsys{PPT}}
\def\mathematica{\scsys{Mathematica}}
\def\css{\scsys{css}}
\def\xslt{\scsys{xslt}}
\def\html{\scsys{html}}
\def\xhtml{\scsys{xhtml}}
\def\openmath{\scsys{OpenMath}}
\def\activemath{\scsys{ActiveMath}}
\def\cpoint{\scsys{CPoint}}
\def\cmathml{\scsys{CMathML}}
\def\pmathml{\scsys{PMathML}}
\def\mathml{\scsys{MathML}}
\def\omdoc{OMDoc}
\def\latexml{\scsys{LaTeXML}}
\def\texfourht{\scsys{TeX4HT}}
\def\perl{\scsys{Perl}}
\def\connexions{\scsys{Connexions}}
\def\cnxml{\scsys{CnXML}}
\def\excdot#1#2{\exists #1.#2}
\def\allcdot#1#2{\forall #1.#2}

\title{Transforming {\TeX/\LaTeX} into XML}
\author{Michael Kohlhase\\Computer Science\\
  International University
  Bremen\\{\tiny\url{http://www.faculty.iu-bremen.de/mkohlhase}}
\and
Bruce Miller\\
DLMF\\
NIST}

\begin{document}
\maketitle

\begin{abstract}
  \begin{oldpart}{extend this by LaTeXML}
  We present a collection of {\TeX} macro packages that allow to markup
  {\TeX/\LaTeX} documents semantically without leaving the document format,
  essentially turning {\TeX/\LaTeX} into a document format for mathematical
  knowledge management (MKM).
  
  We analyze the current practice of semi-semantic markup in {\LaTeX} documents
  and supply a definition mechanism for semantic macros and a non-standard scoping
  construct for them, which is oriented at the semantic depency relation rather
  than the document structure.
  
  We evaluate the {\stex} macro collection on a large case study: the course
  materials of a a two-semester course in Computer Science was annotated
  semantically and converted to the {\omdoc} MKM format Bruce Miller's {\latexml}
  system.
\end{oldpart}
\end{abstract}

\section{Introduction}\label{sec:intro}

\begin{newpart}{reworked the intro, this is still too much content oriented. We need to
    talk about presentation as well. }
The last few years have seen the emergence of various content-oriented {\xml}-based,
markup languages for mathematics on the web, e.g.  {\openmath}~\cite{BusCapCar:2oms04},
{\cmathml}~\cite{CarIon:MathML03}, or our own {\omdoc}~\cite{Kohlhase:omdoc1.2}. These
representation languages for mathematics, that make the structure of the mathematical
knowledge in a document explicit enough that machines can operate on it. Other examples of
content-oriented formats for mathematics include the various logic-based languages found
in automated reasoning tools (see~\cite{RobVor:hoar01} for an overview), program
specification languages (see e.g.~\cite{Bergstra:as89}).

The promise if these content-oriented approaches is that various tasks involved in ``doing
mathematics'' (e.g. search, navigation, cross-referencing, quality control, user-adaptive
presentation, proving, simulation) can be machine-supported, and thus the working
mathematician is relieved to do what humans can still do infinitely better than machines:
The creative part of mathematics --- inventing interesting mathematical objects,
conjecturing about their properties and coming up with creative ideas for proving these
conjectures. However, before these promises can be delivered upon (there is even a
conference series~\cite{MKM-IG-Meetings:web} studying ``Mathematical Knowledge Management
(MKM)''), large bodies of mathematical knowledge have to be converted into content form.

Even though {\mathml} is viewed by most as the coming standard for representing
mathematics on the web and in scientific publications, it has not not fully taken off in
practice. One of the reasons for that may be that the technical communities that need
high-quality methods for publishing mathematics already have an established method which
yields excellent results: the {\TeX/\LaTeX} system: and a large part of mathematical
knowledge is prepared in the form of {\TeX}/{\LaTeX} documents.

{\TeX}~\cite{Knuth:ttb84} is a document presentation format that combines complex
page-description primitives with a powerful macro-expansion facility, which is utilized in
{\LaTeX} (essentially a set of {\TeX} macro packages, see~\cite{Lamport:ladps94}) to
achieve more content-oriented markup that can be adapted to particular tastes via
specialized document styles. It is safe to say that {\LaTeX} largely restricts content
markup to the document structure\footnote{supplying macros e.g. for sections, paragraphs,
  theorems, definitions, etc.}, and graphics, leaving the user with the presentational
{\TeX} primitives for mathematical formulae. Therefore, even though {\LaTeX} goes a great
step into the direction of an MKM format, it is not, as it lacks infrastructure for
marking up the functional structure of formulae and mathematical statements, and their
dependence on and contribution to the mathematical context.

\subsection{The {\xml} vs. {\TeX/\LaTeX} Formats and Workflows}

{\mathml} is an {\xml}-based markup format for mathematical formulae, it is standardized
by the World Wide Web Consortium in {\cite{CarIon:MathML03}}, and is supported by the
major browsers. The {\mathml} format comes in two integrated components: presentation
{\mathml}\twin{presentation}{MathML} and content {\mathml}\twin{content}{MathML}. The
former provides a comprehensive set of layout primitives for presenting the visual
appearance of mathematical formulae, and the second one the functional/logical structure
of the conveyed mathematical objects. For all practical concerns, presentation {\mathml}
is equivalent to the math mode of {\TeX}. The text mode facilitates of {\TeX} (and the
multitude of {\LaTeX} classes) are relegated to other {\xml} formats, which embed
{\mathml}.
 
The programming language constructs of {\TeX} (i.e. the macro definition
facilities\footnote{We count the parser manipulation facilities of {\TeX}, e.g. category
  code changes into the programming facilities as well, these are of course impossible for
  {\mathml}, since it is bound to {\xml} syntax.}) are relegated to the {\xml}
transformation language {\xslt}~\cite{Deach:exls99,Kay:xslt} or proper {\xml}-enabled
programming languages that can be used to develop language extensions.

The {\xml}-based syntax and the separation of the presentational-, functional- and
programming/extensibility concerns in {\mathml} has some distinct advantages over the
integrated approach in {\TeX/\LaTeX} on the services side: {\mathml} gives us better
\begin{itemize}
\item integration with web-based publishing,
\item accessibility to disabled persons, e.g. (well-written) {\mathml} contains enough
  structural information to supports screen readers.
\item reusability, searchabiliby and integration with mathematical software systems
  (e.g. copy-and-paste to computer algebra systems), and
\item validation and plausibility checking.
\end{itemize}
 
On the other hand, {\TeX/\LaTeX}/s adaptable syntax and tightly integrated programming
features within has distinct advantages on the authoring side:
 
\begin{itemize}
\item The {\TeX/\LaTeX} syntax is much more compact than {\mathml} (see the difference in
  Figures~\ref{fig:mathml-sum} and~\ref{fig:mathml-eip}), and if needed, the community
  develops {\LaTeX} packages that supply new functionality in with a succinct and intuitive
  syntax.
\item The user can define ad-hoc abbreviations and bind them to new control sequences to
  structure the source code.
\item The {\TeX/\LaTeX} community has a vast collection of language extensions and best
  practice examples for every conceivable publication purpose and an established and very
  active developer community that supports these.
\item There is a host of software systems centered around the {\TeX/\LaTeX} language that
  make authoring content easier: many editors have special modes for {\LaTeX}, there are
  spelling/style/grammar checkers, transformers to other markup formats, etc.
\end{itemize}
 
In other words, the technical community is is heavily invested in the whole
{\index*{workflow}}, and technical know-how about the format permeates the
community. Since all of this would need to be re-established for a {\mathml}-based
workflow, the technical community is slow to take up {\mathml} over {\TeX/\LaTeX}, even in
light of the advantages detailed above.
 
\subsection{A {\LaTeX}-based Workflow for {\xml}-based Mathematical Documents}
 
An elegant way of sidestepping most of the problems inherent in transitioning from a
{\LaTeX}-based to an {\xml}-based workflow is to combine both and take advantage of the
respective advantages.
 
The key ingredient in this approach is a system that can transform {\TeX\LaTeX} documents
to their corresponding {\xml}-based counterparts. That way, {\xml}-documents can be
authored and prototyped in the {\LaTeX} workflow, and transformed to {\xml} for
publication and added-value services, combining the two workflows.
 
There are various attempts to solve the {\TeX/\LaTeX} to {\xml} transformation problem; the
most mature is probably Bruce Miller's {\latexml} system~\cite{Miller:latexml}. It
consists of two parts: a re-implementation of the {\TeX} {\index*{analyzer}} with all of
it's intricacies, and a extensible {\xml} emitter (the component that assembles the output
of the parser). Since the {\LaTeX} style files are (ultimately) programmed in {\TeX}, the
{\TeX} analyzer can handle all {\TeX} extensions, including all of {\LaTeX}. Thus the
{\latexml} parser can handle all of {\TeX/\LaTeX}, if the emitter is extensible, which is
guaranteed by the {\latexml} binding language: To transform a {\TeX/\LaTeX} document to a
given {\xml} format, all {\TeX} extensions\footnote{i.e. all macros, environments, and
  syntax extensions used int the source document} must have ``{\latexml}
bindings''\index{LaTeXML}{binding}, i.e. a directive to the {\latexml} emitter that
specifies the target representation in {\xml}.
\end{newpart}

\begin{oldpart}{this has to go somewhere}

One of the great problems of mathematical knowledge management (MKM) systems is to
obtain access to a sufficiently large corpus of mathematical knowledge to allow
the management/search/navigation techniques developed by the community to display
their strength. Such systems usually expect the mathematical knowledge they
operate on in the form of semantically enhanced documents.

We will use the term {\defemph{MKM format}} for a content-oriented representation language
for mathematics, that makes the structure of the mathematical knowledge in a document
explicit enough that machines can operate on it. Examples of MKM formats include the
various logic-based languages found in automated reasoning tools (see~\cite{RobVor:hoar01}
for an overview), program specification languages (see e.g.~\cite{Bergstra:as89}), and the
various {\xml}-based, content-oriented markup languages for mathematics on the web, e.g.
{\openmath}~\cite{BusCapCar:2oms04}, {\cmathml}~\cite{CarIon:MathML03}, or our own
{\omdoc} (see {\mysecref{omdoc}}).

In this paper, we will investigate how we can use the macro language of {\TeX} to
make it into an MKM format by supplying specialized macro packages, which will
enable the author to add semantic information to the document in a way that does
not change the visual appearance\footnote{However, semantic annotation will make
  the author more aware of the functional structure of the document and thus may
  in fact entice the author to use presentation in a more consistent way than she
  would usually have.}. We speak of {\twindef{semantic}{preloading}} for this
process and call our collection of macro packages {\stex} (Semantic {\TeX}). Thus,
{\stex} can serve as a conceptual interface between the document author and MKM
systems: Technically, the semantically preloaded {\LaTeX} documents are
transformed into the (usually {\xml}-based) MKM representation formats, but
conceptually, the ability to semantically annotate the source document is
sufficient.

Concretely, we will present the {\stex} macro packages together with a case study,
where we semantically preload the course materials for a two-semester course in
Computer Science at International University Bremen and transform them to the
{\omdoc} MKM format (see section~\ref{sec:omdoc}) with the {\latexml} system (see
section~\ref{sec:latexml}), so that they can be used in the {\activemath}
system~\cite{activemathAIEDJ01}.  For this case study, we have added {\latexml}
bindings for the {\stex} macros, and a post-processor for the {\omdoc} language,
but the {\stex} package should in principle be independent of these two choices,
since it only supplies a general interface for semantic annotation in
{\TeX}/{\LaTeX}. Furthermore, we have semantically preloaded the {\LaTeX} sources
for the course slides (380 slides, 8200 lies of {\LaTeX} code with 336kb). Almost
all examples in this paper come from this case study.
\end{oldpart}
%%% Local Variables: 
%%% mode: stex
%%% TeX-master: "main"
%%% End: 

\section{Tools for {\TeX}/{\LaTeX} to {\xml} Conversion}

The conversion of {\TeX/\LaTeX} documents has been supported by various tools, of
which none has been totally satisfactory yet. For an overview and a test suite
see~\cite{MathmlTeXSuite}.

The {\sc{MMiSSLaTeX}}~\cite{DweLinLuth:tucm04} to {\omdoc} converter is based on a
set of {\LaTeX} styles~\cite[Chapter 4]{Kohlhase:omfmd01} that use {\TeX}'s file
output facility to generate {\omdoc} files directly.
 
The {\connexions} project has developed a {\LaTeX}
package~\cite{WillHenBar:clsfflc03} facilitating content markup in {\LaTeX} for
the {\LaTeX}-to-{\cnxml} conversion, but does not have a transformation tool that
makes use of it.

Romeo Anghelache's {\hermes}~\cite{Anghelache:hermes} and Eitan Gurari's {\texfourht}
systems use special {\TeX} macros to seed the {\tt{dvi}} file generated by {\TeX}
with semantic information.  The {\tt{dvi}} file is then parsed by a custom parser
to recover the text and semantic traces which are then combined to form the output
{\xml} document. While {\hermes} attempts to recover as much of the mathematical
formulae as {\cmathml}, it has to revert to {\pmathml} where it does not have
semantic information. {\texfourht} directly aims for {\pmathml}.

The latter two systems rely on the {\TeX} parser for dealing with the intricacies
of the {\TeX} macro language (e.g. {\TeX} allows to change the tokenization (via
``catcodes'')and the grammar at run-time). In contrast to this, Bruce Miller's
{\latexml}~\cite{Miller:latexml} system and the {\scsys{SGLR}/\scsys{Elan4}}
system~\cite{VanDenBrandStuber2003} re-implement a parser for a large fragment of
the {\TeX} language. This has the distinct advantage that we can control the
parsing process: We want to expand abbreviative macros and recursively work on the
resulting token sequence, while we want to directly translate semantic macros,
since they directly correspond to the content representations we want to obtain.
The {\latexml} and {\scsys{SGLR}/\scsys{Elan4}} systems allow us to do just this.

In the conversion experiment that drove the development of the {\stex} package, we
chose the {\latexml} system, whose {\LaTeX} parser seems to have larger coverage.
We assume that a solution based on the {\scsys{SGLR}/\scsys{Elan4}} system would
work similarly.  Eventually, a combination of both systems might be the way to go:
the {\scsys{Elan4}} system could be integrated into post-processing phase of the
{\latexml} work-flow.  Systems like {\hermes} or {\texfourht} could be used with
{\stex}, given suitable {\stex} bindings provided we find a way to distinguish
semantic- from abbreviative macros.

%%% Local Variables: 
%%% mode: stex
%%% TeX-master: "main"
%%% End: 

% /=====================================================================\ %
% |  latexml.sty                                                        | %
% | Style file for latexml documents                                    | %
% |=====================================================================| %
% | Part of LaTeXML:                                                    | %
% |  Public domain software, produced as part of work done by the       | %
% |  United States Government & not subject to copyright in the US.     | %
% |---------------------------------------------------------------------| %
% | Bruce Miller <bruce.miller@nist.gov>                        %_%     | %
% | http://dlmf.nist.gov/LaTeXML/                              (o o)    | %
% \=========================================================ooo==U==ooo=/ %

% NOTE: Figure out where this should go.
%  At least should define various `semantic enhancement' macros that
% authors using latexml might want.
% But, be careful not to step on the toes of other packages (naming scheme),
% Nor, to assume to much about what semantics authors might want.

\providecommand{\lxDocumentID}[1]{}%

% NOTE: Am I stepping on toes by including these here?
% Common markup junk for LaTeXML docs.
\providecommand{\XML}{\textsc{xml}}%
\providecommand{\SGML}{\textsc{sgml}}%
\providecommand{\HTML}{\textsc{html}}%
\providecommand{\XHTML}{\textsc{xhtml}}%
\providecommand{\MathML}{MathML}%
\providecommand{\OpenMath}{OpenMath}%

% The LaTeXML Logo.
\DeclareRobustCommand{\LaTeXML}{L\kern-.36em%
        {\sbox\z@ T%
         \vbox to\ht\z@{\hbox{\check@mathfonts
                              \fontsize\sf@size\z@
                              \math@fontsfalse\selectfont
                              A}%
                        \vss}%
        }%
        \kern-.15em%
%        T\kern-.1667em\lower.5ex\hbox{E}\kern-.125em\relax
%        {\tt XML}}
        T\kern-.1667em\lower.4ex\hbox{E}\kern-0.05em\relax
        {\sc xml}}%

% Math definining macro.
% Define a math function such that the TeX output is what you might
% expect, while providing the semantic hooks for generating useful xml.

% \lxMathDef{cmd}[nargs][optional]{expansion}[semanticprops]
\providecommand{\lxMathDef}{\lx@mathdef}
\def\lx@mathdef#1{%
  \@ifnextchar[{\lx@mathdef@a{#1}}{\lx@mathdef@a{#1}[]}}
\def\lx@mathdef@a#1[#2]{%
  \@ifnextchar[{\lx@mathdef@opt{#1}[#2]}{\lx@mathdef@noopt{#1}[#2]}}
\def\lx@mathdef@opt#1[#2][#3]#4{%
  \providecommand{#1}[#2][#3]{#4}%
  \@ifnextchar[{\lx@@skipopt}{}}
\def\lx@mathdef@noopt#1[#2]#3{%
  \providecommand{#1}[#2]{#3}%
  \@ifnextchar[{\lx@@skipopt}{}}
\def\lx@@skipopt[#1]{}

% \lxDeclare[declarations]{match}
\newcommand{\lxDeclare}[2][]{}%
% \LxDeclRef{label}
\newcommand{\lxRefDeclaration}[1]{}%
% \LxMathDef{\cs}[nargs][optargs]{presentation}[declarations]
\def\lxDefMath#1{\@ifnextchar[{\LxDefMath@}{\LxDefMath@[]}}%
\def\lxDefMath@[#1]{\@ifnextchar[{\LxDefMath@@}{\LxDefMath@@[]}}%
\def\lxDefMath@@[#1]#2{\@ifnextchar[{\LxDefMath@@@}{}}%
\def\lxDefMath@@@[#1]{}%

% NOTE: It would be good to incorporate Scoping into this macro.
% As defined, it obeys TeX's usual grouping scope.
% However, scoping by `module' (M.Kohlhase's approach) and/or
% `document' scoping could be useful.

% In module scoping, the definition is only available within a
% module environment that defines it, AND in other module envs
% that `use' it.

% In document scoping, the definition would only be available within
% the current sectional unit.  I'm not sure the best way to achieve this 
% within latex, itself, but have ideas about latexml...
% But, perhaps it is only the declarative aspects that are important to
% latexml...

% Expose other declarative macros
\def\LXMID#1#2{\expandafter\gdef\csname xmarg#1\endcsname{#2}\csname xmarg#1\endcsname}
\def\LXMRef#1{\csname xmarg#1\endcsname}
\providecommand{\lxFcn}[1]{#1}
\providecommand{\lxID}[1]{#1}
\providecommand{\lxPunct}[1]{#1}


\baseURI[.]{http://example.com}
\begin{omgroup}[id=sec.reals]{Real Numbers}
\begin{module}[id=reals]
\symdef{RealNumbers}{\mathbb{R}}
\symdef{absval}[1]{\mixfixi[p=2000]|{#1}|}
\symdef{rfrac}[2]{\frac{#1}{#2}}

\begin{definition}[id=reals.def] 
 We denote the set of {\defii{real}{numbers}} we all know and love with
  $\RealNumbers$. 
\end{definition}

\begin{definition}[id=absval.def]
  The absolute value $\absval{r}$ of a real number $r$.
\end{definition}
\end{module}
\end{omgroup}
%%% Local Variables: 
%%% mode: LaTeX
%%% TeX-master: "all"
%%% End: 

\section{Case Studies}
\subsection{Mathematical Document Classes}

\subsubsection{Connexions Modules}

{\cnxlatex} is a collection of {\LaTeX} macros that allow to write {\connexions} modules
without leaving the {\LaTeX} workflow. Modules are authored in {\cnxlatex} using only a
text editor, transformed to PDF and proofread as usual. In particular, the {\LaTeX}
workflow is independent of having access to the {\connexions} system, which makes
{\cnxlatex} attractive for the initial version of single-author modules.


For publication, {\cnxlatex} modules are transformed to {\cnxml} via the {\latexml}
translator and can be uploaded to the {\connexions} system.

\subsubsection{OMDoc Documents}

The \verb|omdoc| package provides an infrastructure that allows to markup {\omdoc}
documents in {\LaTeX}. It provides \verb|omdoc.cls|, a class with the and \verb|omdocdoc.sty|

\subsubsection{Slides and Presentations}

We present a document class from which we can generate both course slides and course
notes in a transparent way. Furthermore, we present a set of {\latexml} bindings for
these, so that we can also generate {\omdoc}-based course materials, e.g. for
inclusion in the {\activemath} system.


%%% Local Variables: 
%%% mode: stex
%%% TeX-master: "main"
%%% End: 

\section{Conclusion and Future Work}\label{sec:concl}

We have presented a system {\stex} of macro-packages for semantic annotation of
{\TeX/\LaTeX} documents and an extension of the {\latexml} system with bindings
for the {\stex} package. This allows us to semantically pre-load {\TeX/\LaTeX}
documents and transform them into {\xml} and ultimately into the {\omdoc} format.

The system is being tested on a first-year computer science course at
International University Bremen. The next case study will be the {\omdoc} 1.2
report~\cite{Kohlhase:omdoc1.2}.

In essence, the {\stex} package together with its {\latexml} bindings forms an
invasive editor for {\omdoc} in the sense discussed in~\cite{KohKoh:cdad04}: The
author can stay in her accustomed work-flow; in the case of {\TeX/\LaTeX}, she can
use the preferred text editor to ``program'' documents consisting of text,
formulae, and control sequences for macros. The documents can even be presented in
by the {\TeX} formatter in the usual way. Only with the semantic preloading, they
can be interpreted as MKM formats that contain the necessary semantic information,
and can even be transformed into explicit MKM formats like {\omdoc}.

Thus the {\stex}/{\latexml} combination extends the available invasive editors
for {\omdoc} to three ({\cpoint}~\cite{KohKoh:cdad04} and
{\sc{nb2omdoc}}~\cite{Sutner:cmnto04} being those for {\ppt} and {\mathematica}).
This covers the paradigmatic examples of scientific document creation formats
(with the exception of MS Word a possible porting target of the VBA-based
application {\cpoint}).

In the future, we plan to extend the system with a {\cnxml} back-end, which
produces the input format for {\connexions} and e-learning document management
system at Rice University. In particular, we will include the {\LaTeX}
package~\cite{WillHenBar:clsfflc03} developed there.


\subsection*{Acknowledgments} This work has profited significantly from
discussions with Bruce Miller and Ioan Sucan. The former is the author of the
{\latexml} system and has extended his system readily to meet the new demands from
our project. The latter is a student a IUB who has faithfully carried the brunt of
the editing load involved with semantic pre-loading the {\LaTeX} slides. Finally I
am indebted to David Carlisle who helped me with the non-trivial hacking involved
in getting the modules to work.

\begin{oldpart}{another conclusion, use parts of this}
\subsection{Conclusion}\label{sec:concl}

The {\stex} collection provides a set of semantic macros that extends the familiar and
time-tried {\LaTeX} workflow in academics until the last step of Internet publication of
the material. For instance, a {\connexions} module can be authored and maintained in
{\LaTeX} using a simple text editor, a process most academics in technical subjects are
well familiar with. Only in a last publishing step (which is fully automatic) does it get
transformed into the {\xml} world, which is unfamiliar to most academics. 

Thus, {\stex} can serve as a conceptual interface between the document author and MKM
systems: Technically, the semantically preloaded {\LaTeX} documents are transformed into
the (usually {\xml}-based) MKM representation formats, but conceptually, the ability to
semantically annotate the source document is sufficient.

 
The {\stex} macro packages have been validated together with a case
study~\cite{Kohlhase:smtl05}, where we semantically preload the course materials for a
two-semester course in Computer Science at International University Bremen and transform
them to the {\omdoc} MKM format, so that they can be used in the {\activemath}
system~\cite{activemathAIEDJ01}. Another study of converting {\LaTeX} materials for the
{\connexions} project is under way.\ednote{say some more}
\end{oldpart}

\begin{oldpart}{use somewhere}
  The nice thing about the infrastructure is that you can still run {\LaTeX} over the
  first form and get the same formula in the DVI file that you would have gotten from
  running it over the second form. That means, if the author is prepared to write the
  mathematical formulae a little differently in her {\LaTeX} sources, then she can use
  them in {\xml} and {\LaTeX} at the same time.
\end{oldpart}

%%% Local Variables: 
%%% mode: stex
%%% TeX-master: "main"
%%% End: 


\begin{small}
\bibliographystyle{alpha}
\bibliography{kwarc,omdoc}
\end{small}
\ednotemessage
\end{document}

%%% Local Variables: 
%%% mode: stex
%%% TeX-master: t
%%% End: 

